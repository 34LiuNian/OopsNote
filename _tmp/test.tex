\documentclass[UTF8]{ctexart}

% ========== 页面与行距 ==========
\usepackage[a4paper,
  left=2.5cm,
  right=2.5cm,
  top=2.5cm,
  bottom=2.5cm
]{geometry}

\usepackage{setspace}
\setstretch{1.25}

\setlength{\parindent}{2em}
\setlength{\parskip}{0pt}

% ========== 字体 ==========
\setCJKmainfont{SimSun}   % 正文宋体
\usepackage{newtxtext,newtxmath}

% ========== 数学 ==========
\usepackage{amsmath,amssymb}

% ========== 枚举与题号 ==========
\usepackage{enumitem}
\setlist[enumerate,1]{
  label=\arabic*.,
  leftmargin=0pt,
  itemindent=2em,
  itemsep=6pt,
  topsep=6pt
}

% ========== 选择题选项 ==========
\newcommand{\choice}[4]{
\begin{tabular}{@{}p{0.45\textwidth}p{0.45\textwidth}@{}}
A.~#1 & B.~#2\\
C.~#3 & D.~#4
\end{tabular}
}

% ========== 文档开始 ==========
\begin{document}
\zihao{-4}

% ======== 试卷头 ========
\begin{center}
{\zihao{-2}\bfseries 2026年普通高等学校招生全国统一考试}\\[6pt]
{\zihao{3} 数学}\\[10pt]
{\zihao{-4}(考试时间:120分钟\quad 满分:150分)}
\end{center}

\vspace{10pt}

% ======== 注意事项 ========
\noindent
\textbf{注意事项:}
\begin{enumerate}[label=(\arabic*),leftmargin=0pt,itemindent=2em,itemsep=0pt]
\item 答卷前,考生务必将自己的姓名、准考证号填写在答题卡上。
\item 回答选择题时,选出每小题答案后,用2B铅笔把答题卡上对应题目的答案标号涂黑。
\item 非选择题用黑色字迹的签字笔在答题卡上作答。
\end{enumerate}

\vspace{10pt}

% ======== 一、选择题 ========
\section*{一、选择题:本题共12小题,每小题5分,共60分。在每小题给出的四个选项中,只有一项是符合题目要求的。}

\begin{enumerate}
\item 已知集合 $A=\{x\mid x^2-3x<0\}$,则集合 $A$ 等于
\choice{$(0,3)$}{$(1,3)$}{$(0,1)$}{$(1,+\infty)$}

\item 已知复数 $z=1+i$,则 $|z|=$
\choice{$\sqrt{2}$}{$2$}{$1$}{$\sqrt{3}$}

\item 设函数 $f(x)=x^2-2x+1$,则其最小值为
\choice{$0$}{$1$}{$-1$}{$2$}
\end{enumerate}

% ======== 二、填空题 ========
\section*{二、填空题:本题共4小题,每小题5分,共20分。}

\begin{enumerate}
\item 已知函数 $f(x)=\sin x$,则 $f(\pi)=$ \underline{\hspace{3cm}}。

\item 椭圆 $\frac{x^2}{4}+\frac{y^2}{3}=1$ 的离心率为 \underline{\hspace{3cm}}。
\end{enumerate}

% ======== 三、解答题 ========
\section*{三、解答题:本题共6小题,共70分。解答应写出必要的文字说明、证明过程或演算步骤。}

\begin{enumerate}
\item(10分)已知等差数列 $\{a_n\}$ 的首项为 $1$,公差为 $2$。
\begin{enumerate}[label=(\arabic*),itemsep=4pt]
\item 求 $a_n$ 的通项公式;
\item 求前 $n$ 项和 $S_n$。
\end{enumerate}

\item(12分)已知函数
\[
f(x)=x^3-3x^2+2.
\]
\begin{enumerate}[label=(\arabic*),itemsep=4pt]
\item 求函数的单调区间;
\item 求函数的极值。
\end{enumerate}
\end{enumerate}

\end{document}
